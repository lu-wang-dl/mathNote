\subsection{Basic facts of addition and subtraction 加减法基础}
\begin{paracol}{2}
These basic strategies can help you remember addition facts for small numbers.
\begin{enumerate}
\item Count on from the greater number. For example, $6+3$. Count on from $6$ 3 times: $7,8,9$. So $6+3=9$.
\item Use doubles. For example, $8+7$. Since $7+7=14$, $8+7 = 14+1=15$.
\item Use 10 to add 8 or 9. For example, $9+5 = 10+4 = 14$.
\end{enumerate}
\switchcolumn[1]
对于小数的加法,有如下的技巧可以帮助我们:
\begin{enumerate}
\item 从大数开始数数、例如: $6+3$。从$6$开始数三次:$7,8,9$.所以$6+3=9$.
\item 翻倍。例如:$8+7$。因为$7+7=14$,$8+7=7+7+1=15$。
\item 利用10做8或9相关的加法。例如:$9+5=10+4=14$。
\end{enumerate}
\switchcolumn[0]*
These basic strategies can help you remember subtraction facts for small numbers.
\begin{enumerate}
\item Count back from the greater number. For example, $6-2$. Count back from $6$ twice: $5,4$. So $6-2=4$.
\item Count up from the lesser number. For example, $7-5$. Count on from $5$ to $7$: $6,7$. So $7-5=2$.
\item Use 10 to subtract 8 or 9. For example, $16-9 = 16-10+1 = 7$.
\end{enumerate}
\switchcolumn[1]
对于小数的减法,有如下的技巧可以帮助我们:
\begin{enumerate}
\item 从大数倒数。例如, $6-2$. 从 $6$ 倒数两个数: $5,4$. 所以 $6-2=4$。
\item 从小数数。例如,$7-5$. 从 5 数到 7: $6,7$. 所以 $7-5=2$。
\item 利用10计算减 8 或 9. 例如, $16-9 = 16-10+1 = 7$。
\end{enumerate}
\switchcolumn[0]*
For the addition and subtraction of large numbers, the column method can be useful. The way to do the column method is as follows:
\begin{description}
\item [Step 1: ] Create columns for each place value. From right to left, we will create a ones column,  tens column, hundreds column, and so on. And we align the numbers by the place value. For example, ones of the first number will align with ones of the second. 
\item [Step 2: ]  Add/subtract the digits in each column starting from ones. 
\item [Step 3: ] Borrowing/ Carrying numbers. If the sum of the digits is 2-digit number, we have to carry a number to the next column. If the top digit in any of the columns is smaller than the bottom digit then we have to borrow from the next column.
\end{description}
\switchcolumn[1]
对于较大的数的加减法,我们会利用竖式来帮助我们计算。其方法如下:
\begin{description}
\item [第1步: ] 列竖式。数的每一位对其。例如第一个数的个位和第二个数的个位要对其。
\item [第2步:] 从个位起,诸位数字相加。
\item [Step 3: ] 处理进位/借位。加法中如果某位数字加和大于$10$,则要向下一位进位。减法中如果某位数字被减数较小,则需要向上一位借位。
\end{description}
\end{paracol}
\begin{example}
238+126
\end{example}
\begin{solution}
$$
\begin{array}{lccc}
&hundreds&tens&ones\\
&\text{百位}& \text{十位}&\text{个位}\\
&2&3&8\\
+&1&2&6\\
\hline
&&&
\end{array}
\Rightarrow
\begin{array}{lccc}
&hundreds&tens&ones\\
&\text{百位}& \text{十位}&\text{个位}\\
&2&3&{\color{red} \bm8}\\
+&1&\ \  2_{{\color{red} \bm 1}}&{\color{red} \bm6}\\
\hline
&&&{\color{red} \bm 4}
\end{array}
$$
$$
\Rightarrow
\begin{array}{lccc}
&hundreds&tens&ones\\
&\text{百位}& \text{十位}&\text{个位}\\
&2&{\color{red} \bm 3}&8\\
+&1&\ \  {\color{red} \bm 2_{1}}&6\\
\hline
&&{\color{red} \bm 6}&4
\end{array}
\Rightarrow
\begin{array}{lccc}
&hundreds&tens&ones\\
&\text{百位}& \text{十位}&\text{个位}\\
&{\color{red} \bm2}& 3&8\\
+& {\color{red} \bm 1}&\ \ 2_{1}&6\\
\hline
&{\color{red} \bm 3}& 6&4
\end{array}
$$
$238+126 = 364.$
\end{solution}
\begin{example}
126-109
\end{example}
\begin{solution}
$$
\begin{array}{lccc}
&hundreds&tens&ones\\
&\text{百位}& \text{十位}&\text{个位}\\
&&&\\
&1&2&6\\
-&1&0&9\\
\hline
&&&
\end{array}
\Rightarrow
\begin{array}{lccc}
&hundreds&tens&ones\\
&\text{百位}& \text{十位}&\text{个位}\\
 & & {\color{red}\bm 1}&{\color{red} 16}\\
&1&\bcancel{{\color{red}\bm2}}&\bcancel{\color{red}\bm 6}\\
-&1&0&9\\
\hline
&&&{\color{red}\bm 7}
\end{array}
$$
$$
\Rightarrow
\begin{array}{lccc}
&hundreds&tens&ones\\
&\text{百位}& \text{十位}&\text{个位}\\
 & & {\color{red}\bm 1}&16\\
&1&\bcancel{{\color{red}\bm2}}&\bcancel{6}\\
-&1&{\color{red}\bm 0}&9\\
\hline
&&{\color{red}\bm 1}&7
\end{array}
\Rightarrow
\begin{array}{lccc}
&hundreds&tens&ones\\
&\text{百位}& \text{十位}&\text{个位}\\
 & & 1&16\\
&{\color{red}\bm 1}&\bcancel{2}&\bcancel{6}\\
-&{\color{red}\bm 1}&0&9\\
\hline
&\bcancel{\color{red}\bm 0}&1&7
\end{array}
$$
$126-109 = 17.$
\end{solution}
   \newpage