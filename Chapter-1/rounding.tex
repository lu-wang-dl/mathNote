\subsection{Rounding 四舍五入}
\begin{paracol}{2}
Rounding means making a number simpler but keeping its value close to what it was. The result is less accurate, but easier to use. 
\switchcolumn[1]
近似是寻找一个和原来的数接近的看起来比较“简单”的数。它不精确,但是相对简单。
\switchcolumn[0]*
There are several different methods for rounding. Here we look at the common method, the one used by most people.
\switchcolumn[1]
有许多种不同的近似方法,通常我们用的方法是四舍五入。其方法如下:
\end{paracol}

\begin{newalg}[How to Round Numbers 四舍五入方法]
\begin{enumerate}
\item Decide which is the last digit to keep
\item Leave it the same if the next digit is less than 5 (this is called rounding down)
\item But increase it by 1 if the next digit is 5 or more (this is called rounding up)
\end{enumerate}

\begin{enumerate}
\item 决定要保持的位
\item  如果其下一位小于5,保持这位的数字(向下近似)
\item  如果其下一位大于等于5,则这位数字加1(向上近似)
\end{enumerate}
\end{newalg}

\begin{example}
Round 73 to the nearest 10 把73近似到十位
\end{example}
\begin{solution}
\begin{paracol}{2}
Since $7$ is the tens position, we want to keep it. The next digit is $3$, which is less than $5$. Therefore, no change is needed. The answer is $70$.
\switchcolumn[1]
因为十位上是 $7$,这是我们要保持的数。而其个位为$3$,小于$5$。所以十位不用加一就是 $7$。所以,四舍五入到十位是 $70$。
\end{paracol}
\end{solution}

\begin{example}
Round 191 to the nearest 100 把$191$近似到百位
\end{example}
\begin{solution}
\begin{paracol}{2}
Since $1$ is the hundreds position, we want to keep it. The next digit is $9$, which is greater than $5$. Therefore, increase $1$ to $2$. The answer is $200$.
\switchcolumn[1]
因为百位上是 $1$,这是我们要保持的数。而其十位为$9$,大于$5$。所以百位加一就是 $2$。所以,四舍五入到百位是 $200$。
\end{paracol}
\end{solution}
\newpage