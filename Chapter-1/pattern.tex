\subsection{Number patterns 数字的规律}
\begin{paracol}{2}
Patterns are all around us! Finding and understanding patterns gives us great power. With patterns we can learn to predict the future, discover new things and better understand the world around us. Number pattern is a list of numbers that follow a certain sequence or pattern.
\switchcolumn
生活中处处有规律存在。能够找到并理解其中的规律能够帮助我们处理很多事情。利用这些规律可以帮助我们进行预测分析,也能够帮助我们更好的理解这个世界。数字的规律就是给出一系列有一定的规律的数字。
\end{paracol}


\begin{example}
Look at the following number sequence and fill in the missing gaps 观察下面的数列,找出规律并填下缺少的数字。
$$
1)\quad 1, 2, 3, 4, 5, \underline{\quad}, \underline{\quad} \quad\quad\quad \quad\quad\quad 2)\quad 2, 4, 6, \underline{\quad}, 10, \underline{\quad}, \underline{\quad}
$$
$$
3)\quad 2, 4, 8, \underline{\quad}, 32, \underline{\quad}, 128 \quad\quad\quad\quad\  4)\quad 1, 2, \underline{\quad}, 7, 11,  \underline{\quad}, 22
$$
$$
5)\quad 1600,\underline{\quad} , 400, 200, \underline{\quad}, 50, \underline{\quad}\quad\ 6)\quad 0, 15, \underline{\quad}, 0, 15, 30 \ \  
$$
\end{example}
\begin{solution}
\begin{paracol}{2}
1) Pattern: Addition of one.
\switchcolumn
规律:下一个是前一个数字加一。
\end{paracol}
$$
1, 2, 3, 4, 5, \underline{{\color{red}\ 6\ }}, \underline{{\color{red}\ 7\ }}
$$
\begin{paracol}{2}
2) Pattern: Addition of two.
\switchcolumn
规律:下一个是前一个数字加二。
\end{paracol}
$$
2, 4, 6, \underline{{\color{red}\ 8\ }}, 10, \underline{{\color{red}\ 12\ }}, \underline{{\color{red}\ 14\ }}
$$
\begin{paracol}{2}
3) Pattern: Doubling each time.
\switchcolumn
规律:下一个是前一个数字的两倍。
\end{paracol}
$$
2, 4, 8, \underline{{\color{red}\ 16\ }}, 32, \underline{{\color{red}\ 64\ }}, 128
$$
\begin{paracol}{2}
4) Pattern: Adding one more each time.(Fibonacci Numbers)
\switchcolumn
规律:相邻两个数字的差逐次加一。(斐波那切数列)
\end{paracol}
$$
1, 2, \underline{{\color{red}\ 4\ }}, 7, 11,  \underline{{\color{red}\ 16\ }}, 22
$$
\begin{paracol}{2}
5) Pattern: Halving each time. 
\switchcolumn
规律:下一个是前一个数字的一半。
\end{paracol}
$$
1600,\underline{{\color{red}\ 800\ }} , 400, 200, \underline{{\color{red}\ 100\ }}, 50, \underline{{\color{red}\ 25\ }}
$$
\begin{paracol}{2}
6) Pattern: Repeated pattern.
\switchcolumn
规律:三个数一组重复出现。
\end{paracol}
$$
0, 15, \underline{{\color{red}\ 30\ }}, 0, 15, 30
$$
\end{solution}

\subsection{Common number sequences 常见的一些数列}
\begin{paracol}{2}
An {\bf Arithmetic Sequence} is a sequence that the difference between one term and the next is a constant. The difference between the terms is called common difference.
\switchcolumn
等差数列:如果一个数列从第二项起,每一项与它的前一项的差等于同一个常数,这个数列就叫做等差数列,而这个常数叫做等差数列的公差。
\end{paracol}
\begin{newprop}[通项公式]
We can write an Arithmetic Sequence as a rule:
$$
\text{k-th term} = \text{first term} + \text{Common difference}\times(k-1)
$$
$$
\text{第几项} = \text{首项} + \text{公差}\times(\text{项数}-1)
$$
\end{newprop}

\begin{newprop}[项数公式]
The number of terms of the Arithmetic Sequence as can be computed by 
$$
\text{Number of terms} = (\text{last term} - \text{first term}) \div \text{Common difference} + 1
$$
$$
\text{项数} = (\text{末项} - \text{首项}) \div \text{公差} + 1
$$
\end{newprop}

\begin{newprop}[Summing an Arithmetic Series 求和公式]
To sum up the terms of this arithmetic sequence, use this formula
$$
\text{sum} = (\text{fisrt term} + \text{last term}) \times \text{Number of terms} \div 2
$$
$$
\text{总和} = (\text{首项} + \text{末项}) \times \text{项数} \div 2
$$
\end{newprop}

\begin{paracol}{2}
An {\bf Geometric Sequence} each term is found by multiplying the previous term by a constant. The difference between the terms is called common ratio.
\switchcolumn
等比数列:从第二项起,每一项与前一项的比都是一个常数,这个数列就叫做等比数列,而这个常数叫做等差数列的公比。
\end{paracol}

\begin{paracol}{2}
The {\bf  Fibonacci Sequence} is found by adding the two numbers before it together. 
\switchcolumn
斐波那契数列:数列从第三项开始,每一项都等于前两项之和。
\end{paracol}
\newpage
