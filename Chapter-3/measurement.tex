\subsection{Measurement 测度}
\paragraph{Time  时间}
\ \\ 
\begin{paracol}{2}
A clock is a device or instrument for measuring or displaying the current time. 
\switchcolumn
钟表是测量时间的基本工具。
\end{paracol}
%\begin{figure}[!hbtp]

\begin{figure}[!hbtp]
\centering
\tikz \node [scale=0.5, inner sep=0] {
\clock{7}{45}
};
\caption{7:45A.M.}
\end{figure}

\begin{paracol}{2}
There are 12 numbers on the clock. It seperates to 5 parts between two numbers.  The short hand is the hour hand. It moves from one number to the next in 60 minutes. The long hand is the minute hand. It mores from one number to the next in 5 minutes. The minute hand moves from one small mark to the next in 1 minute. Sometimes, there are also second hand on the clock.
\switchcolumn
钟表的圆周被12个数码分成12个相等的大格。每个大格又分成5个相等的小格。时针走1个大格是1小时;分针走1个大格是5分钟,1个小格是1分钟。有时钟表上还会有秒针。
\end{paracol}

\begin{paracol}{2}
The relationship between different time units is as follows.
\switchcolumn
不同的时间单位换算如下:
\end{paracol}

\begin{table}[!hbtp]
\begin{center}
\begin{tabular}{rlrl}
60 seconds &= 1 minute, & 60秒 &= 1分\\
60 minutes &= 1 hour, & 60分 &= 1小时\\
24 hours &= 1 day, & 24小时 &= 1 天\\
7  days &= 1 week, & 7天 &= 1周 (1星期)\\
12 months &= 1 year, & 12个月 &= 1年\\
365 days &= 1 year, & 365天 &= 1年(平年)\\
366 days &= 1 leap year, & 365天 &= 1润年\\
10 years &= 1 decade, & 10年 &= 1 十年\\
100 years &= 1 century, & 100年 &= 1世纪
\end{tabular}
\end{center}
\end{table}

\paragraph{Length 长度}
\ \\ 
\begin{paracol}{2}
The metric units of length are as follows.
\switchcolumn
不同的长度单位换算如下:
\end{paracol}

\begin{table}[!hbtp]
\begin{center}
\begin{tabular}{rlrl}
10 mm (millimeter) &= 1 cm (centimeter), & 10毫米 &= 1厘米\\
10 cm (centimeter) &= 1 dm (decimeter), & 10厘米 &= 1分米\\
10 dm(decimeter) &= 1 m (meter), &  10分米 &= 1米\\
1000 m (meter) &= 1 km (kilometer), & 1000米 &= 1千米(公里)\\
1 ft. (Foot) &= 12 in. (inch), & 1 英尺 &= 12 英寸\\
1 in. (inch) &= 2.54 cm (centimeter), & 1英寸 &= 2.54 厘米\\
1 ft. (Foot) &= 0.3048 m (meter), & 1 英尺 &= 0.3048 米\\
3 ft. (Foot) &= 1 yd (yard), & 3英尺 &= 1码\\
1 mi (Mile) &= 1760 yd (yard), & 1英里 &= 1760码
\end{tabular}
\end{center}
\end{table}

\paragraph{Weight 重量}
\ \\ 

\begin{paracol}{2}
The metric units of weight are as follows.
\switchcolumn
不同的重量单位换算如下:
\end{paracol}

\begin{table}[!hbtp]
\begin{center}
\begin{tabular}{rlrl}
1000 g (gram) &= 1 kg (kilogram), & 1000克 &= 1千克(公斤)\\
1000 kg (kilogram) &= 1 T(tonne), & 1000千克 &= 1吨\\
28.3495 g (gram) &= 1 oz (ounce), & 28.3495克 &= 1盎司(安士)\\
16 oz (ounce) &= 1 lb (pound), & 16 盎司 &= 1 磅\\
2000 lb (pound) &= 1 ton, & 2000磅 &= 1英吨
\end{tabular}
\end{center}
\end{table}

\paragraph{Volumn 体积}
\ \\ \ 

\begin{paracol}{2}
The metric units of volumn are as follows.
\switchcolumn
不同的体积单位换算如下:
\end{paracol}

\begin{table}[!hbtp]
\begin{center}
\begin{tabular}{rlrl}
1000 mL (milliliter) &= 1 L (liter), & 1000毫升 &= 1升\\
237 mL (milliliter) &= 1 c. (cup), & 237毫升 &= 1杯\\
2 c. (cup) &= 1 pt. (pint), & 2杯 &= 1品脱\\
2 pt. (pint) &= 1 qt. (quart), & 2品脱 &=1 夸脱\\
4 qt. (quart) &= 1 gal. (gallon), & 4夸脱 &= 1加仑
\end{tabular}
\end{center}
\end{table}

\paragraph{Temperature 温度}
\ \\ 

\begin{paracol}{2}
Temperature is a physical quantity expressing hot and cold. Temperature is measured with a thermometer, historically calibrated in various temperature scales and units of measurement. The most commonly used scales are the Celsius scale, denoted in $^\circ \mathrm{C}$(informally, degrees centigrade), the Fahrenheit scale ($^\circ \mathrm{F}$), and the Kelvin scale.
\switchcolumn
温度是表示物体冷热程度的物理量,而用来量度物体温度数值的标尺叫温度计。目前国际上用得较多的温标有摄氏温标($^\circ \mathrm{C}$()、华氏温标($^\circ \mathrm{F}$)和热力学温标(K)。
\end{paracol}

$$
[^\circ \mathrm{F}] = [^\circ \mathrm{C}]\times \dfrac{9}{5} + 32, \quad  [^\circ \mathrm{C}] = ([^\circ \mathrm{F}]-32)\times \dfrac{5}{9}.
$$

$$
[K] = [^\circ \mathrm{C}] +273.15,\quad [^\circ \mathrm{C}] = [K] - 273.15.
$$
\newpage
