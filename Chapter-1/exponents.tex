\subsection{Exponentiation 指数运算}
\begin{paracol}{2}
Exponentiation is a special operation for writing a product in which all of the numbers being multiplied are the same. For example, $2\times 2\times 2\times 2$ can be written as $2^4$. We call the entire expression $2^4$ a {\bf power}, specifically the power of $2$. The number on the bottom is the base. The number on top is the exponent. 
\switchcolumn[1]
指数运算是多个相同数字的乘积的另一种表达方式。例如:$2\times 2\times 2\times 2$ 可以写成 $2^4$。$2^4$ 被称为指数运算,或2的幂。指数运算中,在下面的数字被称为底数,在上面的数字被称为指数。
\switchcolumn[0]*
When the exponent is $2$, we call it a square which means the product of a number ans itself. For example, $4^2$ is called "four squared". When the exponent is $3$, we call it a cube. So $6^3$ is called "six cubed".
\switchcolumn[1]
当指数为$2$时,我们称为平方。一个数的平方是此数与它的本身相乘所得的乘积。例如:$4^2$称为4的平方。当指数为$3$时,我们称为立方。如$6^3$称为6的立方。
\end{paracol}

\begin{example}
\begin{tabular}{cccc}
a. $10^2$& b. $3^5$ & c. $5^6$ & d. $10^7$
\end{tabular}
\end{example}
\begin{solution}
\begin{align*}
a.\quad & 10^2 = 10\times 10 = 100.\\
b.\quad & 3^5 = 3\times 3\times 3\times 3\times 3 = 243.\\
c.\quad & 5^6 = 5\times 5\times 5\times 5\times 5\times 5 = 125\times 125 = 15625.\\
d.\quad & 10^7 = 10,000,000\\
\end{align*}
\end{solution}

\begin{newprop}[Product of powers 幂相乘]
{\bf Same exponent}: The product of two powers with the same exponent is the product of the two base with the same exponent. For example, \\
同指数幂相乘,指数不变,底数相乘。例如:
$$
3^3\times 4^3 = (3\times 4)^3
$$
{\bf Same base:} The exponent of the product of two powers with the same base is the sum of the two powers. The base is the same.\\
同底数幂相乘,底数不变,指数相加
$$
4^3\times 4^5 = 4^{3+5}
$$
\end{newprop}

\begin{example}
Explain why 
$$
 3^3\times 4^3 = (3\times 4)^3, \quad 4^3\times 4^5 = 4^{3+5}
$$
\end{example}
\begin{solution}
\begin{align*}
&3^3\times 4^3 \\
=& (3\times 3\times 3) \times (4\times 4\times 4)\\
=& (3\times 4) \times (3\times 4) \times (3\times 4)\\
=& (3\times 4)^3
\end{align*}
\begin{align*}
&4^3\times 4^5\\
=& (4\times 4\times 4) \times (4\times 4\times 4\times 4\times 4)\\
=& 4\times 4\times 4 \times 4\times 4\times 4\times 4\times 4\\
=& 4^8\\
=& 4^{3+5}
\end{align*}
\end{solution}

\begin{newprop}[Quotient of powers 幂相除]
{\bf Same exponent}: The quotient of two powers with the same exponent is the quotient of the two base with the same exponent. For example, \\
同指数幂相除,指数不变,底数相除。例如:
$$
6^3\div 2^3 = (6\div 2)^3
$$
{\bf Same base:} The exponent of the quotient of two powers with the same base is the subtraction of the two powers. The base is the same.\\
同底数幂相除,底数不变,指数相减。例如:
$$
4^5\div 4^3 = 4^{5-3}
$$
\end{newprop}

\begin{example}
Explain why 
$$
6^3\div 2^3 = (6\div 2)^3, \quad 4^5\div 4^3 = 4^{5-3}
$$
\end{example}
\begin{solution}
\begin{align*}
&6^3\div 2^3  \\
=& (6\times 6\times 6) \div (2\times 2\times 2)\\
=& 6\times 6\times 6 \div 2\div 2\div 2\\
=& (6\div 2) \times (6\div 2) \times (6\div 2)\\
=& (6\div 2)^3
\end{align*}
\begin{align*}
&4^5\div 4^3 \\
=& (4\times 4\times 4\times 4\times 4)\div (4\times 4\times 4)\\
=& 4\times 4\times 4 \times 4\times 4\div 4\div 4\div 4\\
=& 4^2\\
=& 4^{5-3}
\end{align*}
\end{solution}

\begin{newprop}[Power of powers 幂的幂]
When raising a power to a power in an exponential expression, you find the new power by multiplying the two powers together. For example, \\
一个数的幂的幂等于这两个幂相乘。例如:
$$
(6^3)^3 = 6^{3\times 3}
$$
\end{newprop}

\begin{example}
Simplify. Rewrite the expression in the form $5^{*}$. 化简。把下面表达式写成 $5^{*}$的形式。
$$
(5^3)^2
$$
\end{example}
\begin{solution}
$$
(5^3)^2 = 5^{3\times 2} = 5^6.
$$

\end{solution}\begin{newprop}[Zero-Exponent Rule 零次幂]
Any number raised to the zero power is 1. For example, \\
任何数的零次幂等于1。例如:
$$
181^0 = 1.
$$
\end{newprop}
\newpage
