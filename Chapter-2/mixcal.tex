\subsection{Arithmetic 四则混合运算}

\begin{paracol}{2}
Like whole numbers, decimals and fractions obey the commutative, associative, and distributive laws, and the rule against division by zero.
\switchcolumn[1]
像整数一样,小数和分数的四则运算满足交换律结合律和交换律,以及不能除以零。
\switchcolumn[0]*
In the calculation, if the fractions can transfer to finite decimals, do the calculation after transferring all fractions to decimals. Otherwise, do the calculation after transferring all decimals to fractions. 
\switchcolumn[1]
在小数和分数的四则混合运算中,如果分数能化成有限小数,通常把分数化成有限小数后再计算,如果分数不能化成有限小数,就把小数化成分数后再计算。
\switchcolumn[0]*
If there is division by fractions, transfer the division to multiplication first, then do the calculation.
\switchcolumn[1]
分数乘除混合运算时,一般先把除法变成乘法,然后进行计算。
\end{paracol}

\begin{example}
$4\dfrac{3}{4} - 0.74 + 2\dfrac{3}{5}$
\end{example}
\begin{solution}
\begin{align*}
& 4\dfrac{3}{4} - 0.74 + 2\dfrac{3}{5}\\
= & 4.75 - 0.74 + 2.6\\
= & 4.01 + 2.6\\
= & 6.61
\end{align*}
\end{solution}

\begin{example}
$
3\dfrac{2}{3} + 4.2 -2 \dfrac{1}{7}
$
\end{example}
\begin{solution}
\begin{align*}
&3\dfrac{2}{3} + 4.2 -2 \dfrac{1}{7} \\ 
= & 3\dfrac{2}{3} + 4\dfrac{1}{5} -2 \dfrac{1}{7} \\
= & 7\dfrac{13}{15} -2 \dfrac{1}{7}\\
= & 5\dfrac{76}{105}
\end{align*}
\end{solution}

\begin{example}
$\dfrac{6}{5}\div\dfrac{6}{25}\times \dfrac{1}{4}$
\end{example}
\begin{solution}
\begin{align*}
& \dfrac{6}{5}\div\dfrac{6}{25}\times \dfrac{1}{4}\\
= & \dfrac{6}{5}\times\dfrac{25}{6}\times \dfrac{1}{4}\\
= &  \dfrac{6\times 25\times 1}{5\times 6\times 4}\\
= & \dfrac{5}{4}
\end{align*}
\end{solution}
   \newpage